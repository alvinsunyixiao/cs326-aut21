\documentclass[10pt, twocolumn]{article}

\usepackage[utf8]{inputenc}
\usepackage[english]{babel}
\usepackage[hmargin=.75in, vmargin=1in]{geometry}

\usepackage{amsmath}
\usepackage{amssymb}
\usepackage{amsthm}
\usepackage{bm}

\title{\vspace{-3.0em}Paper Review: Segmentation via Manipulation}
\author{Alvin Sun}

\begin{document}
\maketitle

\section{Paper Summary}
This paper presents a interactive manipulation method that can adjust
grasp plans on the fly based on sensory feedback from interacting with
the environments. This paper focuses particularly on the domain of decomposing
heaps of partially observable objects in a post office settings. The heap of
objects is observed using a real-time range image sensor and represented
as a growing graph with edges indicating spatial relation among the objects.
The control strategy is formulated as a non-deterministic finite-state
Turing machine where the state transitions involves actuation and sensing
that modifies the object graph.
This paper proposed four specific state machines with progressively higher
level of interaction, carried out experiments
for each of the strategies, and recorded the success rates. The results indicated
that the interactive strategy is the most effective for this heap sorting task.

\section{What I Learned}
\begin{enumerate}
  \item The concept of using interaction to progressively build up the understanding
    of a partially observable scene is really useful, especially when the scene
    is dynamic, where interaction changes the observability of the objects of interests.

  \item Feedback control is the way to go for any type of control task, as it
    adapts to environmental changes that is almost certainly going to occur in
    reality.
\end{enumerate}

\section{Opinions}

\subsection{Up Votes}
I agree about how this work sets up the experiments with strategies derived
with progressively higher level of interaction. The experimental results perfectly
validate their hypothesis that interaction can be very effective for heap segmentation
tasks.

\subsection{Down Votes}
Given that this work is carried out in the last century when just compute is
a scarce resource, I believe it is quite amazing how much this paper accomplishes.
However, one slight disagreement I have is how the evaluation uses state transition
number to compare across the different strategies. Given that the different
strategies use different set of actions for state transitions, I don't think
the number of transitions is a good metric for representing the grasp plan efficiency.
A more appropriate metric could be the average time to complete the sorting
since all experiments are conducted with the same hardware platform.

\section{Evaluations}
The goal of this paper is to propose a novel paradigm of scene segmentation that
takes advantage of interactive manipulation and interaction. This is a perfectly
valid objective as this work is presented in the pre-deep-learning era, where
pure vision based segmentation methods are intractable due to compute and
resolution bottlenecks. The domain of this work also presents possibilities
for real-world deployment as it does not require too much a priori assumptions
on the geometries of the objects to be manipulated and segmented.

The overall quality of this work is exceptional in my opinion. Given such
limited compute and sensing hardware, this paper really show cases how
explicitly exploiting interaction can play an important role in scene
understanding as well as manipulation. This work also clearly states their
assumptions. Even though it is far from what a day-to-day manipulation
complexity a human has to deal with, those assumptions are quite realistic
in controlled domains such as an assembly line or post office as mentioned
in the paper. Under those assumptions, the experimental results also demonstrated
very high success rate given the scene is composed of partially observable
overlapping objects with unknown dynamics. Again I would like to emphasize
how impressive this is given the limited compute and sensing resource they
had at the time.

\section{Questions}
\begin{enumerate}
  \item In strategy 4, if the partially visible objects are first displaced
    out of the scene, wouldn't that make them lost objects?

  \item In the evaluation metrics, what does the number of
    top-most surface segments identified per scene mean?
\end{enumerate}

\end{document}
