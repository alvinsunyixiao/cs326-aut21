\documentclass[10pt, twocolumn]{article}

\usepackage[utf8]{inputenc}
\usepackage[english]{babel}
\usepackage[hmargin=.75in, vmargin=1in]{geometry}

\usepackage{amsmath}
\usepackage{amssymb}
\usepackage{amsthm}
\usepackage{bm}

\title{\vspace{-2.0em}Paper Review of CHOMP\\Gradient Optimization Techniques for Efficient Motion Planning}
\author{Alvin Sun}

\begin{document}
\maketitle

\section{Paper Summary}
This paper proposed an optimization based method for efficiently planning
collision free trajectories. This method explicitly defines the optimization
objectives that considers both obstacles as well as dynamic requirements.
The obstacle term is formulated as a potential function integrated through
an SDF (signed distance field) representation, while the dynamical requirement
is modeled with a quadratic function over the trajectory. The proposed
regularized optimization method, like Gauss-Newton, converges much faster than
naive Euclidean gradient, but does not require differentiability. They have also
shown many successful implementations of their optimization planner
on a variety of hardware platforms.

\section{What I Learned}
\begin{enumerate}
  \item Signed distance field can be computed super efficiently with
    Euclidean Distance Transform, which takes linear time.
  \item Finite differencing of the trajectory can be constructed with
    appropriate linear operators.
\end{enumerate}

\section{Opinions}

\subsection{Up Votes}
\begin{itemize}
  \item I like how obstacles are formulated into the optimization objective.
    The SDF is both efficient to compute as well as easy to be incorporated
    with objective functions. This eliminates the need for a sampling based
    planner that explicitly checks for collision.
  \item I also like their approach on using local linear approximation with
    regularized optimization objective. This not only results in a efficiently
    computable update rule, but also ensures smoothness across the trajectory
    update.
\end{itemize}

\subsection{Down Votes}
I don't really disagree much with this work. One slight shortcoming of CHOMP
is that it might be hard to explicitly plan for tasks that requires close
proximity or even contacts. The optimization results naturally repel the trajectory
away from any occupied space. It might be hard to plan trajectories that
aim to move towards some target object (obstacle).

\section{Evaluations}
The goal of this paper is to introduce a unified optimal control framework
that is not only able to generate smooth trajectories, but also takes
environment, more specifically obstacles, into account. It is a perfectly valid goal
as previous work on motion planning considering obstacle avoidance mostly
uses sampling based method, which can produce jerky and dynamically unfeasible
trajectories. On the other hand, previous work on optimal control has not been
successfully take obstacle avoidance into account with efficient computing.
The significance of bridging the gap between optimal control and obstacle avoidance
also lies in making the whole robotic control system easier to implement
with such unified framework.

This paper is of very high quality in my opinion. The proposed methodology
provides sound derivation of optimization theories that are both tractable
and efficient to implement in practice. Their experimental results also
suggests a huge success in a variety of sophisticated real world robotic
system including the \textit{LittleDog} from Boston Dynamic Inc. One slight
unaddressed assumption they made is that CHOMP requires the knowledge of
3D environment around the manipulator. Accurate 3D reconstruction is in fact
still a open and challenging problem to be solved.

\section{Questions}
\begin{enumerate}
  \item Where does the metric $M$ in Equation (3) comes from?
  \item Is the computation still tractable if the manipulators are instead
    modeled with polygonal meshes?
\end{enumerate}

\end{document}
