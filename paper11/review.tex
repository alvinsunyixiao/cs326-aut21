\documentclass[10pt, twocolumn]{article}

\usepackage[utf8]{inputenc}
\usepackage[english]{babel}
\usepackage[hmargin=.75in, vmargin=.75in]{geometry}

\usepackage{amsmath}
\usepackage{amssymb}
\usepackage{amsthm}
\usepackage{bm}

\title{\vspace{-3.0em}Paper Review\\DART: Dense Articulated Real-Time Tracking}
\author{Alvin Sun}

\begin{document}
\maketitle

\section{Paper Summary}
This paper proposed a framework, DART, for tracking generic objects in
real-time. This work represents objects as a tree of kinematic constraints between
rigid bodies with pre-defined geometry, which are locally modeled as SDF (signed
distance function). The tracking of the object is done by optimizing for
the kinematic states (such as joint angles and camera poses) between those
rigid bodies.
The tracking algorithm is derived in a Kalman Filter like
fashion, with a process update and a measurement correction. Given the model
of the object to be tracked as well as some initial guess on the kinematic
states, the process model gets to predict what the SDF should look like. Then
a measurement SDF, that is computed based on a depth image, is used to make a
correction through optimizing a symmetric objective function.
The performance of DART is evaluated on a
variety of tasks. The results are comparable or even better than many methods
that are specialized for tracking particular types of objects. The proposed
tracking algorithm is also trivially parallelizable, which means that it can easily
be implemented GPU to achieve real-time performance.

\section{What I Learned}
\begin{enumerate}
  \item Lie algebra can be really useful when dealing with optimization
    on 3D transformations and 3D rotations. The locally linear structure
    has really nice properties for calculus to be applicable in such
    constrained manifold.

  \item The negative information (i.e. empty spaces) in SDFs can also be
    utilized in the optimization problem for finding the most likely
    kinematic states.
\end{enumerate}

\section{Opinions}

\subsection{Up Votes}
\begin{itemize}
  \item I really like how they formulate the measurement correction as optimizing
    for a symmetric function. The objective function can analytically describe
    how likely a kinematic state is considering both the predicted SDF and the
    measured SDF, which both contains come level of noises.

  \item I also like the simple formulation for this tracker, which makes the
    computation for each pixel independent of each other, and thus can be
    parallelized easily.
\end{itemize}

\subsection{Down Votes}
One thing I think this paper might fall short on is probably using a
constant position motion model for the process update. The whole purpose
for tracking is when the scene is dynamic, either the object or the camera
is moving, or both. A constant velocity motion model will make more sense
in the context of dynamic tracking.

\section{Evaluations}
The purpose of this paper is to obtain a framework that can track generic
objects with depth camera sensors in real-time. This is a solid objective
as most of the previous work in the tracking land is developed around
specialized domains such as human-body, hands, simple geometries etc. Few
has proposed framework that is capable of tracking generic objects. DART
does deliver the goal well that it achieves comparable state-of-the-art performance
in previously specialized domains, which makes the algorithm much more
appealing to be extendable to more diverse settings.

The overall quality of this paper is exceptional in my opinion, as their methods
are really generalizable to all kinds of settings. However, one important
assumption for this method to work is that the rigid parts of the objects have
known 3D geometry. For objects such as hand and human body that has relatively
known scale, this could be easy to obtain. However, for other commonly seen
objects in real-life, as simple as cylindrical ones like mugs and cups, we will
need to get a model per objects due to the scale difference. This could be
quite cumbersome if we want to track multiple objects or unknown objects
specified in the image space. So this assumption could be the biggest limiting
factor for this approach to become a anyone-can-use tracker for all settings.
Other then that, one assumption made in the derivation is the independence
between all pixels. This is unrealistic and makes the whole theory a little
less sound then the rest of the derivation. Nonetheless, DART is a great
paper that delivers the objective well by achieving state-of-the-art performance
across multiple domains.

\section{Questions}
\begin{enumerate}
  \item How is inverse Hessian (line 4 of the algorithm) related to the
    covariance of the uncertainty?

  \item Does it method still work if the object model is not scaled correctly?
    (i.e. slightly smaller or bigger than the actual object).
\end{enumerate}

\end{document}
