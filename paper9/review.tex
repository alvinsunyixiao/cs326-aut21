\documentclass[10pt, twocolumn]{article}

\usepackage[utf8]{inputenc}
\usepackage[english]{babel}
\usepackage[hmargin=.75in, vmargin=.75in]{geometry}

\usepackage{amsmath}
\usepackage{amssymb}
\usepackage{amsthm}
\usepackage{bm}

\title{\vspace{-3.0em}Paper Review\\Hierarchical Task and Motion Planning in the Now}
\author{Alvin Sun}

\begin{document}
\maketitle

\section{Paper Summary}
This paper presents a recursively hierarchical planner that integrates both
task level planning and motion level planning. They formulate the planning
problem as meeting a conjunction of conditions, or fluents as they call it.
Each condition might be achieved through achieving its pre-conditions and
some primitive action. The planning process search from the most abstract goal
state condition, and builds up a hierarchical planning tree where
the less abstract sub-goals are generated to meet the pre-conditions. The length
of the planned execution is somewhat minimized by finding the sub-goal with
the weakest pre-conditions (easiest to achieve). The lowest level
action planner uses sampling based method such as RRT to generate the motion
of the robotic manipulator, while tightly integrated with hierarchical
task planner. The author also conducted experiments by hand designing a set
of conditions and primitive actions and the relationship among each other under the setting of
commonly seen indoor cleaning tasks. The results has shown that their
hierarchical planner is capable of planning for complicated multi-sequence
tasks in long horizon.

\section{What I Learned}
\begin{enumerate}
  \item Long-sequence task planning problems can be modeled with hand
    designed hierarchical task dependence, while the plan can be generated
    through search algorithms on such hierarchical dependence tree.

  \item The length of the planned task sequence can be optimized by
    meeting sub-goals with the weakest pre-conditions.
\end{enumerate}

\section{Opinions}

\subsection{Up Votes}
\begin{itemize}
  \item I like how they walk through a concrete design of the hierarchical planner
    they are proposing. The walk through of planning the problem of cleaning
    dishes in the room really made the abstract mathematical definitions of those
    fluents, operators, and goals very easy to follow.

  \item I also like how they provided proofs for the completeness and correctness
    of the proposed hierarchical planner. This is crucial for real-world deployment
    as it provides certain level of guarantees on successfully finding feasible
    plans even for complicated long-sequence tasks.
\end{itemize}

\subsection{Down Votes}
Even though they mentioned in the very beginning that the states are three dimensional,
their illustrations are pretty much two dimensional. I think they over simplified
many real-world geometries such as dishes and dish washers. For example, planning
for grasp pose on dishes and bowls alone is quite a challenging task, but here
they are simplified as simple polygonal objects. Dish washer is abstracted
simply as an area in this paper, but in reality could have complex geometries
to deal with.

\section{Evaluations}
The goal of this paper is to present a planning methods that tightly integrates
task level planning and motion planning. This is a valid objective because
until the time of this work, there has been a long standing challenge for
combining task planners and geometric motion planners and taking the complementary
strength from both types of planners. This work delivers their goal by bringing
forth a formally treated design of hierarchical planning system that
is capable of planning long horizon multiple-sequence tasks.

The overall quality of this paper is great. They did deliver a formal representation
of such hierarchical planning system that can both be implemented and be proven
complete and correct. However, one implicit assumption for this method to work
is that there is some well defined (also fixed) relationship among some pre-defined
sub-goals. Normally, for one specific task such as cleaning dishes with a dish
washer, those sub-goals and their dependency with each other are well defined
and easy to design. For more complex day-to-day routines, different people
may have different preferences, and therefore different things can be done
with different ways and in different orders. It will then become very difficult
to hand engineer the task dependencies for those more complicated situations.
Another thing that this method fall short of is that it assumes the knowledge
of the full states of the world as well as the robot manipulator. This is
to-date still unrealistic given estimation under unknown environment is
still a challenging open questions to be solved.

\section{Questions}
\begin{enumerate}
  \item How do they define the sub-goals with the "weakest" pre-conditions? Is
    it with the least number of pre-conditions?
  \item I am a little confused about what they meant by "postponing" a pre-condition.
\end{enumerate}

\end{document}
