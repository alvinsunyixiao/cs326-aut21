\documentclass[10pt, twocolumn]{article}

\usepackage[utf8]{inputenc}
\usepackage[english]{babel}
\usepackage[hmargin=.75in, vmargin=1in, tmargin=.3in]{geometry}

\usepackage{amsmath}
\usepackage{amssymb}
\usepackage{amsthm}
\usepackage{bm}

\title{Paper Review: Postural Hand Synergies for Tool Use}
\author{Alvin Sun}

\begin{document}
\maketitle

\section{Paper Summary}
This paper presents a study on postural hand synergies with data collected
from human subject trials. The experiment asks each of the 5 subjects to imagine
picking up some commonly seen objects when the posture of the hand grasp 
is recorded in the form of joint angles. A PCA analysis is then carried out 
across the joint angles on the hands, and 
it is found that most of the variances are contained in the first two principle
components. It is also found that different grasps vary on a
continuum of spectrum instead of clustered into discrete groups.

\section{What I Learned}

I learned that given that many degree of freedom a human hand
has, the actual effective dimension is much lower due to correlation between different
joint angles. It is also surprising that the study found no evidence of 
clustering occurs on the grasp of different objects. Intuitively, we make
discrete grasp configuration for different object, but the study somehow shows
that the posture varies on a continuous spectrum.

\section{Opinions}

\subsection{Up Votes}
\begin{itemize}
    \item I like how this paper visualizes raw data in multiple ways. 
        The correlation plot immediately
        reveals relations between different joints. It also verifies the 
        fact that we don't have individual control over some of the finger joints.
    \item I also love the ray-cast rendering of the hand posture in the PCA 
        analysis. It gives direct visual intuition on what are the dominant modes 
        of hand postures.
\end{itemize}

\subsection{Down Votes}
Given most if not all of the joint angles are biologically constrained,
it does not seem like the correlation among them is linear. PCA can 
only account for linear coordinate transformations, and therefore, has 
a limited capability of explaining the underlying dynamics between 
the fingers.

\section{Evaluations}

The goal of this paper is to study the postural hand synergies to 
\begin{enumerate}
    \item provide insights into how human plans for hand grasp configurations;
    \item explore correlations between the different degree of freedom of a human hand;
    \item find out whether hand postures can be clustered or classified for different
        object being grasped.
\end{enumerate}
The overall analysis on the postural hand synergies is quite significant as 
it not only provides direct insights for planning robotic grasp configurations,
but also hints that the control of grasp posture may be regulated independently 
of contact forces.

The overall quality of this paper is solid. The data is collected rigorously
with specialized sensor equipment while the diversity of the data is ensured 
through having multiple subject, multiple trials, and multiple object grasps.
The PCA analysis is definitely one of the simplest and most powerful techniques 
for studying feature correlations. The paper has also shown that the effective 
information of the human grasp can be compressed down to just 2 or 6 dimensions 
depending on the fineness of control. However, one slight 
shortcoming of this approach is the linear assumption that comes with PCA, 
which is almost certainly not true for the model of human finger control.

\section{Questions}
\begin{enumerate}
    \item It seems like the orientation of the hand can be synergized with 
        the overall posture of the hand, why not consider at least the wrist 
        angle of the posture? It seems like the rendered hand postures all 
        have the same top-grasp orientation.
    \item How is the discrimination function computed and how is the confusion 
        matrix (of posture-to-object) generated?
\end{enumerate}

\end{document}