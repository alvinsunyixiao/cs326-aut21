\documentclass[10pt, twocolumn]{article}

\usepackage[utf8]{inputenc}
\usepackage[english]{babel}
\usepackage[hmargin=.75in, vmargin=.75in]{geometry}

\usepackage{amsmath}
\usepackage{amssymb}
\usepackage{amsthm}
\usepackage{bm}

\title{\vspace{-3.0em}Paper Review of Lessons from the Amazon Picking Challenge\\
       Four Aspects of Building Robotic Systems}
\author{Alvin Sun}

\begin{document}
\maketitle

\section{Paper Summary}
This paper discussed four spectrum of robotic system building under the context
of the Amazon Picking Challenge. The four aspects are Modularity vs. Integration,
Computation vs. Embodiment, Planning vs. Feedback, and Generality vs. Assumptions.
The authors first introduces the overall algorithms for their first-place
solution, where they uses feedback guided control of a suction-cup based
manipulator, in combination with their perception module that segments
visual information to determine object locations. They also showed quantitative
evaluation of their grasping success rate both in the contest and in a replayed
setting. Failure cases are thoroughly analyzed and improvements are proposed
with the discussion for the four trade-off spectrum of design choices.

\section{What I Learned}
\begin{enumerate}
  \item Modularity insures a smooth development cycle but imposes difficulties
    for evaluating the robotic systems as a whole. Tightly integrated solutions
    are harder to develop but may yield higher overall performance.

  \item Feedback guided control are really powerful that it might completely
    replace the need for a motion planner as in the case of this Amazon
    Picking Challenge.
\end{enumerate}

\section{Opinions}

\subsection{Up Votes}
\begin{enumerate}
  \item I agree about this author's claim that few researchers in the robotic
    community has tried to explicitly characterize robotic system building.
    This is true especially in the academia where people usually focuses on
    performance of a single module within a robotic system rather than
    the combined system as a whole.

  \item I also strongly agree with their choice with a holonomic mobile base
    which reduces quite a lot of complexity in planning motion with the manipulator.
    This seems like a perfect example for the spectrum between software and hardware,
    and this is just simple hardware that dramatically reduces software complexity.
\end{enumerate}

\subsection{Down Votes}
I don't quite agree one of the four aspects they proposed, which is planning vs.
feedback. Even though they showcase that with good feedback, a need for planner
might be completely eliminated as in their case. However, I don't think these
two things are necessarily trade-off, but rather could be complement of each other.
A good feedback should be able to enhance planning but not replace it.

\section{Evaluations}
This paper aims at discussing multiple aspects of robotics system building and
how design choices can affect the overall system complexity and performance. This
is a perfectly valid objective as few prior work has focused on evaluating
robotic applications from a system level, but they rather put more focus on
the individual sub-problem they are trying to solve. I also agree that the
system level design choices greatly impacts the overall performance, so
it is definitely worth looking into these system building spectrum. The author
also delivers their goal such that they also discussed their proposed
findings under a concrete robotic contest setting with quantitative
evaluation metric.

The overall quality of this paper is excellent, as it not only proposed the four
trade-off spectrum in robotic system building, but also evaluated their own
award winning design choices within those spectrum. The assumption they made
for this specific picking challenge is that objects are placed in a known shelf,
which is quite reasonable given such controlled contest environment. What matters
is their achieved generality under such assumption, which is their robot's
capabilities of handling a variety of uncertainties in the actual challenge.
They have also thoroughly analyzed their failure cases by replaying the challenge
in a lab environment. One slight short coming is that even though they proposed
future improvements within the guidelines of the four aspects, they did not
show any results for the proposed improvements. This might be what's saved for
future directions.

\section{Questions}
\begin{enumerate}
  \item How do they exploit the force feedback to activate the suction cup? Is
    it when some force detected is over a pre-determined threshold?

  \item Why other teams rarely chose to use a holonomic base?
\end{enumerate}

\end{document}
