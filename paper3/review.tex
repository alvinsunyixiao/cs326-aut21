\documentclass[10pt, twocolumn]{article}

\usepackage[utf8]{inputenc}
\usepackage[english]{babel}
\usepackage[hmargin=.75in, vmargin=1in, tmargin=.3in]{geometry}

\usepackage{amsmath}
\usepackage{amssymb}
\usepackage{amsthm}
\usepackage{bm}

\title{Paper Review\\Extrinsic Dexterity: In-Hand Manipulation with External Forces}
\author{Alvin Sun}

\begin{document}
\maketitle

\section{Paper Summary}
This paper exploits a particular type of robotic in-hand manipulation that utilizes
resources outside of the capability of the grasp actuators themselves. This
type of manipulation is then named ``extrinsic dexterity manipulation.'' This
paper specifically categorizes extrinsic dexterity manipulation into three
different types. This work then performed experiments, based on some proposed
grasp graph, on a few primitive shapes
for the different types of extrinsic dexterity. Some qualitative and quantitative
results are reported for the successfulness of hand-tuned open loop control
on those different objects and different situations.

\section{What I Learned}
When we are manipulating objects with robotic grippers, it is crucial to be
aware of the dynamics of both the objects and its surrounding environments.
Simple design of the fingers, like the one used in this paper, can introduce
limited level of manipulation themselves. However, if we takes factors such as
gravity and external obstacles into account, we may be able to do much more than
using just the fingers.

\section{Opinions}

\subsection{Up Votes}
I agree with the idea that extrinsic resources can affect the manipulated object
in much more diverse way then the fingers themselves can. Even thinking about how
human manipulate objects with hands, we take advantage of factors such as weight
distributions and fixed platforms all the time.

\subsection{Down Votes}
\begin{itemize}
  \item All experiments conducted in this paper is hand-coded with open loop
    control. Even though they analyze the sensitivity in initial condition
    error, it is still quite far away from being deployed to real-world situations
    where uncertainties are a major factor in pick-and-place tasks.
  \item This paper only reveals the fact that the tuned control is from trial
    and error but does not discuss the process of their tuning. The reported
    success on the different scenarios might come with the cost of excessive
    number of failing trials.
\end{itemize}

\section{Evaluations}

The goal of this paper is to exploit the effect of extrinsic dexterity for
in-hand manipulation. This is a perfectly valid goal and it is also quite significant
since most studies on grasp manipulations by that time is focusing on the control
of the gripping actuators themselves. Extrinsic dexterity brings in much more
possibilities for what a simple gripper can achieve.

However, I think the overall quality of this work is not so great. There are
rarely any theoretical analysis on their proposed types of grasp. Most importantly,
all the experiments are done with hand-coded open loop controller, meaning that
there is almost certainly no tolerance to any disturbances or changes to the systems.
Although they did some quantitative analysis on how much of an initial condition
variation can be tolerated for certain type of grasp, the tolerance is quite low
(sub centimeter level). Any new object, new gripper configuration, will require
excessive re-tuning for it to work not so reliably. That said, given this was an
early-stage work on the area of extrinsic dexterity, I do want to give them the
credit for bringing the importance of this concept up to the public.

\section{Questions}
\begin{enumerate}
  \item how is Fig. 12 obtained? Are the samples interpolated and then visualized
    with contours?
  \item I am wondering how many trials failed before coming up with an controller
    that can succeed in consecutive 50 tries.
\end{enumerate}

\end{document}
