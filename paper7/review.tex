\documentclass[10pt, twocolumn]{article}

\usepackage[utf8]{inputenc}
\usepackage[english]{babel}
\usepackage[hmargin=.75in, vmargin=.75in]{geometry}

\usepackage{amsmath}
\usepackage{amssymb}
\usepackage{amsthm}
\usepackage{bm}

\title{\vspace{-3.0em}Paper Review\\Is Imitation Learning the Route to Humanoid Robots?}
\author{Alvin Sun}

\begin{document}
\maketitle

\section{Paper Summary}
This paper reviews developments in up-to-date neural sciences that have specifically
been focusing on imitation learning. It also draws connection to how artificial
intelligence has been benefiting from such neurological intuitions. This paper
provides a glossary over the common concepts related to imitation learning,
and went through each one with their relationship to imitation algorithms.
Evidence of "mirror neurons" found in monkeys confirmed some implicit relationship
between observing and imitating, which further confirms the existence of imitation
mechanisms in biological systems. This review also goes through different approaches
to imitation learning as well as each of the relationship to human learning.
Finally, it summarizes with on going challenges yet to be solved including
but not limited to a good representation for action-perception coupling.

\section{What I Learned}
\begin{enumerate}
  \item There are this "mirror system" in our neural system where neurons respond
    both to observation and execution events.
  \item Even for nowadays algorithms, using movement primitives might provide better
    and more interpretable high level motion planning of humanoid robots.
  \item
    Perception and motor control are coupled not only during motor control, but
    also during learning.
\end{enumerate}

\section{Opinions}

\subsection{Up Votes}
\begin{itemize}
  \item I strongly agree that for any humanoid control, the control and perception
    are deeply coupled. Even though this review is published in last century,
    it is still true that there haven't been a generically good learning representation
    for this action-perception coupling.
  \item I also agree on the general approach for drawing intuition from neural
    science researches when developing imitation learning methods. If we are to
    imitate biological systems with robots, we'd better know how biological
    systems learn themselves.
\end{itemize}

\subsection{Down Votes}
Since this is again a survey / review paper, and also given it is published about
20 years ago, I don't really have too much negative opinion
about the paper. One thing I slightly disagree about the paper is the focus on
humanoid robots. I don't think all those neurological concepts brought up
in the paper is limited to aiding the design of just humanoid robots. I believe
that any robots that aims for learning from perception can benefit from such
neurological intuitions.

\section{Evaluations}
The goal of this review paper is to summarize some of the up-to-date development
in neural sciences and how imitation learning algorithms for robotic applications
can benefit from connecting with those neurological concepts. This is a perfectly
valid and novel objective as there wasn't a clear connection between
the ongoing research in imitation learning algorithms and neural psychology
sciences. Introducing imitation learning from both viewpoint of robotics and
cognitive science can certainly open up new directions in the research of
robotic learning. To name a few, the learning movement primitives explicitly
is still an open area in robotic planning today; learning for joint action-perception
representation is also under active research currently.

The overall quality of this paper is good. Since this is also a review paper,
it does not have much of a point to prove. The evaluation is done mostly on
the comprehensiveness of the review. It has done pretty well introducing the
terminologies with neural / cognitive science and drawing connection to
the ongoing research of imitation learning at that time. One slight
short coming of this work is that this review seems to be
conducted under the assumption of building humanoid robots, but in general, I don't
think robot has to be humanoid or even bio-like, I think those cognitive
research findings are applicable to most robotic applications nowadays as well.

\section{Questions}
\begin{enumerate}
  \item Could you given an example for \textbf{accommodation}?
  \item What exactly is \textit{model-based imitation learning}?
\end{enumerate}

\end{document}
