\documentclass[10pt, twocolumn]{article}

\usepackage[utf8]{inputenc}
\usepackage[english]{babel}
\usepackage[hmargin=.75in, vmargin=.75in]{geometry}

\usepackage{amsmath}
\usepackage{amssymb}
\usepackage{amsthm}
\usepackage{bm}

\title{\vspace{-3.0em}Paper Review of GelSight}
\author{Alvin Sun}

\begin{document}
\maketitle

\section{Paper Summary}
This paper introduces a new way of high resolution tactile sensing via
imaging techniques. The proposed GelSight sensors combines a single camera
with a coated elastomer surface that can be illuminated with well controlled
lighting conditions. Using multiple colors and lighting angles combined with
the calibrated reflectance model, the captured image can be converted to
surface normals at each pixel locations, and then ultimately integrates to
a z-height map that fully recovers the geometry of the contacting portion of
the object. In addition to sensing shapes, GelSight is also capable of estimating
contact forces by exploiting the relationship with the motion field on the
elastomer surface. The motion field is captured by tracking a dense grid of
markers added to the elastomer. CNN based vision algorithms is then used to
estimate the forces given marker motions. Slippage detection follows from a
previous work which is by analyzing the statistics of the motion field. This
work not only develops the theory, but also fabricated a family of GelSight
sensors with different configurations. The experiments conducted with the actual
sensors verify that this type of sensor can do what the paper claims with high
fidelity.

\section{What I Learned}
\begin{enumerate}
  \item Given some specific contact shape, the contact forces are almost linear
    in relation to the magnitude of the motion field.

  \item In controlled lighting conditions, we can also do stereo photometric
    reconstruction with a single camera but multiple light sources.

  \item The inverse of nonlinear projection models can usually be found via
    table lookup.
\end{enumerate}

\section{Opinions}

\subsection{Up Votes}
I really like their stereo photometric method for finding the height
map of the surface. It is really an elegant way to do 3D surface reconstruction
with only monocular vision. Under controlled lighting and reflectance, when comparing
to the more traditional multi-camera stereo vision, this could be not only
easier and cheaper to fabricate but also more robust against noises as it
does not require to solve the correspondence problem in stereo vision.

\subsection{Down Votes}
I don't quite agree with their method for force estimation. Even though CNN
has been shown to be really high performing in image related tasks, predicting
contact forces purely from motion fields images, in my opinion, is reliable
enough for it to be deployed into unknown environment. As the authors mentioned,
the shape of the object could influence the slope of the linear relationship between
the force and the magnitude of the motion field. The contact motions are usually
dynamic and time varying, so I believe modeling force estimation with time series
of motion field changes as oppose to static motion field is more appropriate.

\section{Evaluations}
The goal of this paper is to develop a new type of tactile sensor that is
not only capable of outputting high-resolution information about contact
shape and forces, but also is straightforward to fabricate.
This is a perfectly valid objective as very few previous work presents a
tactile sensor design that captures this much of high-resolution details of
each contact. Also, the fabrication details are usually neglected from
previous work, which is usually nontrivial due to the design complexity.
This paper certainly achieves their goal by presenting novel theories of vision
based tactile sensing that also follows from detailed design with fabrication
procedures.

The quality of this paper is really beyond exceptional in my opinion. It is really
the first to make this rich tactile sensing data available for a wide range
of applications. The only theoretical assumption they made is knowing the
lighting conditions as well as the reflectance of the coating material. This
assumption is relatively realistic given that this paper also provided calibration
procedures for finding those prior knowledge. Not only the proposed method
provides rich shape and contact force information, but also is quite feasible
to fabricate into a relatively small-sized device that can be utilized in
a robotic gripper setting. They also show quantitative evaluation results
for real-world data, which further verifies the efficacy of their proposed methods.

\section{Questions}
\begin{enumerate}
  \item If they used CNN for force / torque estimation, why did they not use it
    for slip detection?

  \item I am wondering how this sensor is integrated with optimal planning
    of dexterous manipulation?
\end{enumerate}

\end{document}
