\documentclass[10pt, twocolumn]{article}

\usepackage[utf8]{inputenc}
\usepackage[english]{babel}
\usepackage[hmargin=.75in, vmargin=.75in]{geometry}

\usepackage{amsmath}
\usepackage{amssymb}
\usepackage{amsthm}
\usepackage{bm}

\title{\vspace{-2.0em}Paper Review\\Online Movement Adaptation Based on Previous Sensor Experiences}
\author{Alvin Sun}

\begin{document}
\maketitle

\section{Paper Summary}
This paper presents an online adaptation algorithm that can effectively generate robust
robotic arm control by leveraging sensor readings from previous
successful trial runs. Their method first learns some human demonstrated trajectories
as a DMP (Dynamic Movement Primitives) system. It then uses this learned DMP model
to generate predictive sensor responses in subsequent runs. Using this predictive
model and the actual sensor feedback, they can modify the trajectory by adding
the feedback to the transformation system in the DMP. Fianlly, using this
adapted trajectory plan, a low-level position and force controller is applied
for the robotic arm and fingers to track the generated trajectories. In parallel,
the authors also proposed a mathematically correct DMP formulation that deals
with quaternion trajectories. Their experiments demonstrated that the grasping
can handle much more perturbed initial condition using their proposed adaptation
algorithm.

\section{What I Learned}
\begin{enumerate}
  \item A goal directed trajectory can be efficiently learned with a DMP model that has
    nice guarantees of convergence and stability properties.
  \item Sensory feedback, in additional to the dynamical system itself, can contain
    some predictive information about the executed control.
\end{enumerate}

\section{Opinions}

\subsection{Up Votes}
\begin{itemize}
  \item I strongly agree with their proposed re-formulation of DMP for
    quaternion trajectories. The formulation effectively ensures that the
    derivatives lives in the tangent space of current quaternion state.
    I think this can also be generalized to other constrained manifolds
    such as $SE3$, so that the DMP can jointly consider translation and
    rotation.

  \item I also like how they formulate the robotic arm controller in a way
    that the position control and force control is decoupled, and can be
    simply additive in the overall controller.
\end{itemize}

\subsection{Down Votes}
I don't this work conducted enough experiments to show the effectiveness of
their proposed adaption method. They only showed the successful grasp on
a simple cylinder shaped cup, and a laid-down bottle. The latter experiment
also neglects the contact dynamic between the gripper and the table
due to fear of collision. The demonstrated examples are quite limited
when comparing to real-world demands for robotic grasping tasks.

\section{Evaluations}
The goal of this paper is to exploit the predictive information present
in sensor experiences for robotic grasping problems and to find out how those information
can be converted to useful grasp plans that can be robust against
unforeseen perturbations. This is a perfectly valid objective as few work
before this paper has exploited the fact that successful execution of
robot manipulation can generate some {\it expected} sensor readings that
can be useful for adjusting subsequent executions that deviates from
the expectation. This paper also achieves their objective by demonstrating
how incorporating the adaptation helps improve the control robustness
against initial condition perturbation.

The overall quality of this paper is great. They made a few assumptions for
their proposed method. One of them is that
that some expert demonstration, or some successful grasp execution is
available. Another one is that a successful execution should map
to some expected sensor experience. Under these assumptions, their experiments
show that the system works as expected. However, the assumptions could
be unrealistic if the robotic manipulator has too many DoF such that even
for human it is unintuitive to control. For the second assumption, there might
be multiple different but all successful grasps that can map to drastically
different sensor responses. Under such condition, it becomes unclear about
how to generate feedback trajectory corrections given those different
"correct" sensor experiences.

\section{Questions}
\begin{enumerate}
  \item What is the $J_{sensor}$ in Equation (8)? And how does that relates
    to the intuition of what $K_1$ and $K_2$ means.
  \item Their methods seem to naturally generate this compliant behavior
    given the adaptation. Why are they still concerned that the robotic
    manipulator might collide with the table and break the fingers?
\end{enumerate}

\end{document}
