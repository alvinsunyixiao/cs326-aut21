\documentclass[10pt, twocolumn]{article}

\usepackage[utf8]{inputenc}
\usepackage[english]{babel}
\usepackage[hmargin=.75in, vmargin=1in]{geometry}

\usepackage{amsmath}
\usepackage{amssymb}
\usepackage{amsthm}
\usepackage{bm}

\title{\vspace{-3.0em}Paper Review\\Visual Servo Control Part 1: Basic Approaches}
\author{Alvin Sun}

\begin{document}
\maketitle

\section{Paper Summary}
This tutorial paper introduces two classical visual servo control methods
in great detail: image-based visual servo (IBVS) and position-based visual servo (PBVS).
It first introduces the problem of visual servoing as a minimization of error
based on some image features and camera parameters. Based on such overall
problem formulation, this paper then derives from scratch the velocity controller
that ideally makes the tracking error decay exponentially. To achieve such
controller, the pseudoinverse of the interaction matrix (which relates camera velocity
and rate of error changes) is required. Different approaches for estimating
this pseudoinverse are compared in terms of tracking behavior both in image
space and in camera motion. In addition to tracking feature points in image space
(IBVS), this paper also shows the derivation for a visual servo controller
that directly track camera position(PBVS). Stability analysis is carried out
for both IBVS and PBVS, and results are two-sided. Even though local convergence
is usually guaranteed, IBVS does not guarantee any global asymptotic convergence
due to having multiple local minima. On the other hand, while PBVS shows favorable
convergence property in theory, it is very sensitive to estimation error,
which will affect the stability in practice.

\section{What I Learned}
\begin{enumerate}
  \item There is a trade off between the performances of camera motion
    tracking and image space feature tracking. A straight line trajectory
    is usually desirable in both domain, but not achievable at the same time.

  \item Using a combination of $L_e^+$ and $L_{e^*}^+$ in IBVS can achieve
    more balanced and smooth tracking on both camera domain and image domain.
\end{enumerate}

\section{Opinions}

\subsection{Up Votes}
\begin{itemize}
  \item I like how this tutorial structures the content in a very
    organized way, starting from problem formulation to goals to achieve.
    It first gives a comprehensive review on camera projection and 3D geometries,
    and derives the visual servo control law from scratch. It makes the math
    fairly straight for readers to follow.

  \item I also really like the graphics in this paper. It really aids understanding
    the different type of controllers and performances. The figure that
    shows a geometrical interpretation of IBVS also makes understanding
    the benefit of using a hybrid controller clearer.
\end{itemize}

\subsection{Down Votes}
Among all the nice derivations this tutorial presents, there is one
short coming in my opinion, which is the limited number of examples.
Most if not all examples used in this paper are tracking of four coplanar
points. This does not come nearly as close to any objects in reality. I think
some examples should at least consider using polygonal 3D shaped objects.

\section{Evaluations}
The goal of this paper is to provide a tutorial on the emerging visual servoing
problem as well as comparing different classical methods that attempt to solve
the problem. This is a valid objective as it provides a quick background
introduction of visual servoing
without requiring extensive computer vision and control background for the readers.
It neatly derives the basic approaches to achieve visual servoing while at the
same time sets off a solid background for more advanced methods presented in
part 2 of this tutorial series.

I believe the quality of this paper is exceptional. It comprehensively covers
multiple approaches that make different level of assumptions. For example, if
depth information is available, we can use IBVS with the exact pseudoinverse
of the interaction matrix. If not, this paper also shows that there is another
way of estimating the pseudoinverse by fixing it with the optimal-state
interaction matrix, which is really to implement in practice. The graphics
shown in this paper is also exceptional for understanding the performances
of different controllers in both camera motion tracking and image space
tracking. Finally, it also carried out theoretical stability analysis
for both IBVS and PBVS, which leads to a conclusion and discussion
on the challenging open problems in visual servoing.

\section{Questions}
\begin{enumerate}
  \item I am quite curious why does estimating $L_e^+$ with $L_{e^*}^+$ still
    work when the error is large (e.g. Figure 2)?

  \item In the stability analysis, why is Equation 19 satisfied when the
    approximations involved in $\widehat{L_e^+}$ are not too coarse? What does
    it even mean to have approximations not too coarse?
\end{enumerate}

\end{document}
