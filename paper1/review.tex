\documentclass[10pt, twocolumn]{article}

\usepackage[utf8]{inputenc}
\usepackage[english]{babel}
\usepackage[hmargin=.75in, vmargin=1in, tmargin=.3in]{geometry}

\usepackage{amsmath}
\usepackage{amssymb}
\usepackage{amsthm}
\usepackage{bm}

\title{Paper Review: Planning Optimal Grasps}
\author{Alvin Sun}

\begin{document}
\maketitle

\section{Paper Summary}
This paper presents a formulation for quantifying the quality of different 
grasp configurations on objects with known (parameterized) shapes. The quality 
value represents a lower bound on the maximum wrench that can be resisted 
per unit force exerted by the gripper. The quality value can also be 
geometrically interpreted as the radius of the sphere that is inscribed inside
the set of reachable wrenches. This paper also provides
computationally feasible algorithms for computing such quality values as well
as example algorithms for planning optimal grasps on polygonal objects.

\section{What I Learned}

\begin{description}
    \item[Force Closure.] The state where all forces and moments introduced 
        by the different contact locations are canceled out. It can also be 
        verified by checking if the origin is contained in the set of
        reachable wrenches or not.

    \item[Friction Cone.] Static friction forces, or wrenches to be more 
        general, can be modeled using a geometric cone inside the wrench space, 
        where all wrenches inside the cone results in non-slipping situations.

    \item[Convex Hull.] This operator can be used
        to represent sets of elements that can be obtained by convex combinations
        of the vertex elements.
\end{description}

\section{Opinions}

\subsection{Up Votes}
\begin{itemize}
    \item This paper provides beautiful geometric interpretation of its
        derived theories, i.e. the closest distance from the boundary
        of the set of reachable wrenches to the origin is equivalently 
        the proposed quality value.
    \item I like the design of using a flexible norm definition of the
        generalized force vector. This allows us to optimize for different 
        objectives (maximum force vs. total force).
\end{itemize}

\subsection{Down Votes}
\begin{itemize}
    \item I don't think computing the Minkowski sum is efficient, especially
        when we are using large number of primitive wrench vectors to 
        approximate the friction cone boundary. The number of vertices on
        the convex hull can potentially reach $m^n$.
    \item For non-convex objects such as anything that has a handle, it
        becomes sub-optimal if we plan the grasps based only on the convex
        hull of the object.
    \item The definition of $\|w\| = \sqrt{\|F\|^2 + \lambda \|\tau\|^2}$
        is not very well thought out. Even though the paper adds the $\lambda$
        to account for the scale difference between forces and torques, 
        it is still somewhat unsure how $\lambda$ will affect the overall 
        optimal criteria.
\end{itemize}

\section{Evaluations}

The goal of this paper is to provide a unified formulation of optimal criteria
for grasp related tasks. The paper delivers such formulation nicely with
flexible choices of optimization objectives as well as algorithms that 
quantitatively and feasibly computes for the optimal criteria. The most significant
contribution, in my opinion, is the unified formalization such that both 
independent and combined forces of the grippers can be optimized.

The overall quality of this paper is great, due to its sound derivation of 
the proposed quality value formulation as well as the algorithms with finite
computation. However, this paper does lack a  quantitative evaluation of 
how good the planned grasps are compared to 
previous work. The assumption it makes is also somewhat unrealistic given
real world uncertainties in all kinds of modeling. The proposed methods require
prior knowledge of the exact friction model as well as geometric model of the 
object to be grasped, which is sometimes too complicated to obtain in
real-world situations.

\section{Questions}
\begin{enumerate}
    \item I am a little confused by the definition 
        $B\bm{g} = \{\bm{w}\; |\; \bm{w}A\bm{g} \text{ is true, and } \|\bm{g}\| = 1\}$.
        How does $\bm{g}$ ends up being a constraint on the set (of $\bm{w}$)?

    \item What happens if part of the surfaces of an object is 
        parameterized by some continues curvature (e.g. sphere, ellipsoid, etc.)?

    \item How do you estimate forces exerted by the gripper when the gripper 
        is some complicated shapes with multiple joints?
\end{enumerate}

\end{document}